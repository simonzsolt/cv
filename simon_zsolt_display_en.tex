\documentclass[12pt,oneside,a4paper]{article}

\usepackage[
  top=11em, 
  bottom=11em, 
  left=10em,
  right=10em]
  {geometry}

\usepackage{fancyhdr}
\lhead{\leftHeader}
\rhead{\rightHeader}
\renewcommand{\headrulewidth}{0pt}
\renewcommand{\footrulewidth}{0pt}

\hyphenation{ma-gyar}

\usepackage{hyperref}
\usepackage[dvipsnames]{xcolor}
\hypersetup{
    colorlinks=true,
    urlcolor=MidnightBlue
}

\usepackage{fontspec}
\setmainfont[Mapping=tex-text,Numbers=OldStyle,Ligatures=TeX]{Linux Biolinum O}

\setlength{\parindent}{0em}
\renewcommand{\baselinestretch}{1.1}

\usepackage{graphicx}
\graphicspath{ {./} }
\usepackage{wrapfig}

\setlength{\parindent}{0em}

\def \leftHeader{CV}
\def \rightHeader{Zsolt Simon}
\def \position{JavaScript Engineer}

\setmainfont[Mapping=tex-text,Numbers=OldStyle,Ligatures=TeX]{Linux Biolinum O}

\begin{document}

\begin{wrapfigure}{r}{8em}
  \flushright
  \includegraphics[scale=0.4]{portrait_mono} 
\end{wrapfigure}

\begin{Large}
  \textsc{Zsolt Simon}
  \vspace{1em}
\end{Large}

\large

Date of birth: 24 October 1989\\
Residence: VIII. district, Budapest\\ 
GitHub: \href{https://github.com/simonzsolt}{\textbf{simonzsolt}}\\
E-mail: \href{mailto:simon.zsolt@mail.com}{simon.zsolt@mail.com}\\
Phone: \href{tel:0036204110724}{+36 20 4110 724}

\vspace{1em}

Job applied for: \position

\vspace{1em}

Work experience

\begin{itemize}
  \item{ 
    2017--2018 - Javascript developer fellowship in Digital Humanities at the Zeno Karl Schindler Foundation located at SISMEL (Società Internazionale per lo Studio del Medioevo Latino), Florence. 
    \href
    {http://www.sismelfirenze.it/index.php/formazione/borse-e-premi/item/219-mirabile-zeno-karl-schindler-foundation-fellowship-in-digital-humanities-2018} 
    {[Call].}
  }
  \item{From 2017 - Contributor in ``POSTDATA – Poetry Standardization and Linked Open Data'' project. 
    I presented the datamodell for the database of 16\textsuperscript{th}  century Hungarian poems 
    (Répertoire de la poésie hongroise ancienne, RPHA, 
    \href
      {http://rpha.elte.hu/}
      {[Website]}
    ).}
  \item{7 November 2016 - Presentation at ``Pesti Bölcsész Akadémia'' conference series held at Eötvös Loránd University of Budapest, 
    Faculty of Humanities, \textit{Hálózati versadatbázisok [Online databases of verse]},
    \href
      {http://pestibolcseszakademia.blog.hu/2016/10/04/digitalis\_bolcseszet\_szovegen\_innen\_es\_tul\_marothy\_szilvia\_es\_simon\_zsolt\_kurzusa}
      {[Program].} 
    }
  \item{17 October 2016 - Presentation at ``Pesti Bölcsész Akadémia" conference series held at Eötvös Loránd University of Budapest, 
    Faculty of Humanities, \textit{Gráfadatbázisok az irodalomtudományban [Graph databases in literary studies}],
    \href
      {http://pestibolcseszakademia.blog.hu/2016/10/04/digitalis\_bolcseszet\_szovegen\_innen\_es\_tul\_marothy\_szilvia\_es\_simon\_zsolt\_kurzusa}
      {[Program].} 
    }
  \item{2015–2016 - Technical editor of the journal \textit{Információtörténeti Műhely [Information History Workshop]} 
    for the following editions:}
    \begin{itemize}
      \item{Bognár, Péter \textit{A régi magyar párrímköltészet német vonatkozásai}, 2016,
        \href
          {http://renaissance.elte.hu/wp-content/uploads/2016/02/parrim.pdf}
          {[PDF].}
      }
      \item{Horváth, Iván \textit{Ómagyar szövegemlékek mint textológiai tárgyak}, 2015,
        \href
          {http://renaissance.elte.hu/wp-content/uploads/2016/01/Omagyar\_szo\%CC\%88vegemle\%CC\%81kek\_d-ed-150bb.pdf}
          {[PDF].}
      }
      \item{Tubay, Tiziano \textit{A székely írás kutatásának története}, 2015,
        \href
          {http://renaissance.elte.hu/wp-content/uploads/2016/04/Tubay\_A-szekely-iras-kutatasanak-tortenete\_OSzK\_2015\_digitalis.pdf}
          {[PDF].}
      }
    \end{itemize}
  \item{24 November 2015 - Presentation at the ``DHU2015 Workshop, Számítógép az irodalomtudományban'' 
    [Computers in Literary Studies], held by the Library of the Hungarian Academy of Sciences and Budapest University of Technology and Economics,
  \href
    {http://dhu2015.mit.bme.hu/felhivas} 
    {[Call for papers].}
  }
  \item{2015 - Development of the databse concerning medieval Hungarian book culture known as THECA,
    \href
     {http://hece.elte.hu/index.php/theca/}
     {[Impressum].}
  }
  \item{2015 - Maintaining the WordPress blog: HECE -- Humanism in East Central Europe, 
    \href
      {http://hece.elte.hu/}
      {[About].}
  }
  \item{2014 - Collaborator at the Digital Philology Division at National Széchényi Library.}
  \item{2014 - Co-developer and technical editor of RMeX [Database of Old Hungarian Exempla],
    \href
     {http://sermones.elte.hu/exemplumadatbazis/}
     {[Impressum].}
  }
  \thispagestyle{fancy}
  \item{2013 - Junior editor of the online edition \textit{Mathias Rex 1458--1490: Hungary at the Dawn
      of the Renaissance},
    \href
      {http://renaissance.elte.hu/?page\_id=665} 
      {[Impressum].}
  }
  \item{2013--2016 - Internship at the Department of Early Hungarian Literature, 
    Institute of Hungarian Literature and Cultural Studies 
    at Eötvös Loránd University of Budapest.
  }
\end{itemize}

Studies and training

\begin{itemize}
  \item{From 2015 ongoing studies (Master's degree) at the Center for Renaissance Studies 
    (CHER - Centre des Hautes Études de la Renaissance) at Eötvös Loránd University of Budapest}
  \item{24–28 October 2016 - DARIAH Winter School ``Open Data Citation for Social Science and Humanities'', 
    Humanities at Scale project (HaS), Prague. }
  \item{1--6 February 2016 - ``Paleography, Codicology, Philology.
    Digital Editing of Medieval Manuscripts Training Programme''.
    Klosterneuburg, Vienna. }
  \item{2015 Diploma (Bachelor's degree in Hungarian) at Eötvös Loránd University of Budapest}
\end{itemize}


Technical skills

  \begin{itemize}
    \item{Javascript: ES6, Node.js, Angular}
    \item{SQL and NoSQL databases: MariaDB, SQL Server, 
      MongoDB, OrientDB, ArangoDB, Elasticsearch, Neo4j}
    \item{git, JIRA}
    \item{UNIX-like systems}
    \item{PHP, Perl, \LaTeX}
    \thispagestyle{fancy}
    \item{AWS, Heroku, OpenShift}
  \end{itemize}

Languages
  
  \begin{itemize}
    \item{English: C1 complex certificate of language, fluent in both speaking and writing.}
    \item{Italian: basic reading and listening proficiency.}
    \item{Latin: basic reading proficiency.}
  \end{itemize}

\thispagestyle{fancy}

\end{document}