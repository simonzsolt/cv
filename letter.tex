\documentclass[12pt,oneside,a4paper]{article}

\usepackage[
  top=11em, 
  bottom=11em, 
  left=10em,
  right=10em]
  {geometry}

\usepackage{fancyhdr}
\lhead{\leftHeader}
\rhead{\rightHeader}
\renewcommand{\headrulewidth}{0pt}
\renewcommand{\footrulewidth}{0pt}

\hyphenation{ma-gyar}

\usepackage[hyphens]{url}

\usepackage{hyperref}
\usepackage[dvipsnames]{xcolor}
\hypersetup{
  colorlinks=true,
  urlcolor=MidnightBlue
}

\usepackage{fontspec}
\setmainfont[Mapping=tex-text,Numbers=OldStyle,Ligatures=TeX]{Linux Biolinum O}

\renewcommand{\baselinestretch}{1.1}

\usepackage{graphicx}
\graphicspath{{./}}
\usepackage{wrapfig}

\setmainfont[Mapping=tex-text,Numbers=OldStyle,Ligatures=TeX]{Linux Biolinum O}

\setlength{\parindent}{1.5em}
\renewcommand{\baselinestretch}{1.1}
\def \addressee {Bene Studio}

\begin{document}
	

	\thispagestyle{empty}
		
	Tisztelt \addressee!

	\vspace{1em}

	Kérem fogadják a jelentkezésemet a DreamJobson meghirdetett Node.js Informatikai mérnök pozícióra. 
	A hirdetésben leginkább a jó munkakörnye\-zet és a fejlődési lehetőségek vonzottak a jelentkezésre.

	Az utóbbi években talán nem szokatlan, hogy más területen végzett szakemberek helyezkednek el informatikai állásokban.
	Az én esetemben egyetemi tanulmányaim elejétől fogva jellemző volt, hogy programozói munkát is végzek kutatómunkám mellett.
	Az itthon egyre kevésbé megtűrt bölcsészeti-informatika lenne a pontos megnevezése a tanulmányaimank, ha még létezne ilyen szak.
	Régi magyar szövegek kutatásának elősegítésére vállaltam szerepet különböző szövegkiadások és adatbázisok fejlesztésében. 
	Mesterszakos szakdolgozatomat 16-ik századi magyar versek adatmodellezéséről írtam.
	Idén októberben fejeztem be egy éves projektet a firenzei SISMEL intézetben, ahol többek között egy web crawler fejlesztése volt a célkitűzés, egy százezres nagyságrendű URL állományhoz.

	Az ilyen és ehhez hasonló projektekből szereztem tapasztalatot alkalmazások csapatban való fejlesztésében, tesztelésében és üzemeltetésében.
	Ugyanakkor tapasztaltam a projektirányítás fontosságát, gyakran az ide\-álistól távol eső körülmények között.

	Kész vagyok egy modern, versenyképes csapat aktív résztvevője lenni, keresem a kihívásokat és a lehetőségeket, hogy fejlesszem a tudásom.

	\begin{flushleft}
		Üdvözlettel,

		Simon Zsolt
	\end{flushleft}

\end{document}