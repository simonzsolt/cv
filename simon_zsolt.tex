\documentclass{article}

\usepackage{hyperref}

\begin{document}

CV

Simon Zsolt 

GitHub: \href{https://github.com/simonzsolt}{simonzsolt}

Születési dátum: 1989 OKtóber 24

Cím: Magyarország, 2600, Vác, Erzsébet utca 10/A

E-mail: simon.zsolt@mail.com

Telefon: +36 20 4110 724

Tanulmányok

\begin{itemize}

  \item{ 2015-től Eötvös Loránd Tudományegyetem Bölcsésztudományi Kar Reneszánsz Tanulmányok mesterszak }

  \item{ 2015: BA Eötvös Loránd Tudományegyetem Bölcsésztudományi Kar magyar alapképzési szak }

\end{itemize}

Tudományos és szakmai tevékenységek

\begin{itemize}

  \item{ 2017-től közreműködő a  ``POSTDATA – Poetry Standardization and Linked
       Open Data'' projektben. }

  \item{ 2016 Október 17: Előadás a Pesti Bölcsész Akadémia konferenciasorozaton (ELTE)
          Gráfadatbáztisok az irodalomtudományban címmel 
          \href{http://pestibolcseszakademia.blog.hu/2016/10/04/digitalis\_bolcseszet\_szovegen\_innen\_es\_tul\_marothy\_szilvia\_es\_simon\_zsolt\_kurzusa}
          {[Program].} }

  \item { 2016 Október 24–28: DARIAH Winter School ``Open Data
          Citation for Social Science and Humanities'', Humanities at Scale project (HaS),
          Prága. }

  \item{ 2016 November 7: Előadás a Pesti Bölcsész Akadémia konferenciasorozaton (ELTE)
        Hálózati versadatbázisok címmel 
        \href
          {http://pestibolcseszakademia.blog.hu/2016/10/04/digitalis\_bolcseszet\_szovegen\_innen\_es\_tul\_marothy\_szilvia\_es\_simon\_zsolt\_kurzusa}
          {[Program].} 
        }

  \item{ 2016 Február 1–6: Részvétel a ``Paleography, Codicology, Philology.
       Digital Editing of Medieval Manuscripts Training Programme'' képzésen.
       Klosterneuburg, Bécs. }

  \item{ 2015–2016: Technikai szerkesztője az Informáctiótörténetti Műhely című sorozat alábbi megjelenéseinek: }

    \begin{itemize}

      \item{ \textsc{Bognár} Péter, \textit{A régti magyar párrímköltészet német vonatkozásati}, 2016,
        \href
          {htp://renaissance.elte.hu/wp-content/uploads/2016/02/parrim.pdf}
          {[PDF].}
      }

      \item{ \textsc{Horváth} Iván, \textit{Ómagyar szövegemlékek mtint textológtiati tárgyak}, 2015,
        \href
          {http://renaissance.elte.hu/wp-content/uploads/2016/01/Omagyar\_szo\%CC\%88vegemle\%CC\%81kek\_d-ed-150bb.pdf}
          {[PDF].}
      }

      \item{ \textsc{Tubay} Tiziano, \textit{A székely írás kutatásának története}, 2015,
        \href
          {http://renaissance.elte.hu/wp-content/uploads/2016/04/Tubay\_A-szekely-iras-kutatasanak-tortenete\_OSzK\_2015\_digitalis.pdf}
          {[PDF].}
      }
      
    \end{itemize}

  \item{ 2015 November 24: Előadás a ``DHU2015 Workshop, Számítógép az
    irodalomtudományban'' című workshopon, az MTA BTK Irodalomtudományi Intézete és a BME Méréstechnikai és Információs Rendszerek Tanszéke közös szervezésével. 
    \href
      {htp://dhu2015.mit.bme.hu/felhivas} 
      {[Felhívás].}
  }

  \item{ 2015: THECA – A középkori magyarországi könyvkatalógusok és könyvjegyzékek adatbázis fejlesztése
    \href
      {http://hece.elte.hu/index.php/theca/}
      {[Impresszum].}
  }

  \item{ 2015: A HECE -- Humanism in East Central Europe WordPress alapú blog szerkesztése. 
    \href
      {http://hece.elte.hu/}
      {[Honlap].}
  }

  \item{ 2014: Az Országos Széchényi Könyvtár Digitális Filológia osztályánal korai közreműködője. 
    Az osztály hosszas szervezést követően nem alakult meg teljesen. }

  \item{ 2014: A Régi Magyar Exemplumadatbáztis társ-fejlesztője és műszaki szerkesztője.
    \href
      {htp://sermones.elte.hu/exemplumadatbazis/}
      {[Impresszum].}
  }

  \item{ 2013: A \textit{Mathtias Rex 1458--1490: Hungary at the Dawn
         of the Renaissance} című kötet Junior szerkesztője.
    \href
      {http://renaissance.elte.hu/?page\_id=665} 
      {[Impresszum].}
  }

  \item{ 2013--2016: Gyakornoki pozíció az ELTE BTK Magyar Irodalom- és Kultúratudományi 
    Intézet Régi Magyar Irodalom Tanszékén.
  }

\end{itemize}

Nyelvtudás
  
  \begin{itemize}

    \item{ Angol: C1 komplex nyelvvizsga, folyékon szövegértés és fogalmazás }

    \item{ Latin: alapszintű olvasási kompetencia }

    \item{ Olasz: alapszintű olvasási és szövegértési kompetencia }
  
  \end{itemize}

Szakmai ismeretek

  \begin{itemize}
  
    \item{ PHP }

    \item{ Javascript:

      \begin{itemize}

        \item{ ES6, Node.js, Angular }

      \end{itemize}
    }

    \item{ git, JIRA }

    \item{ HTTP protokol }

    \item{ SQL és NoSQL adatbázisok: 

      \begin{itemize}
    
        \item{ MariaDB, SQL Server, MongoDB, OrientDB, ArangoDB, Elasticsearch, Neo4j }
    
      \end{itemize}
    }

    \item{ UNIX-típusú rendszrek ismerete }

    \item{\LaTeX}

    \item{ Perl }

  \end{itemize}

      % // other
      % 26–28 April 2017 - Participation in the "Late Medieval and Early Modern Book
      %  Culture - Book Production, Book Ownership, and Book Collecting"- 5th
      %  TRAINING SCHOOL COST ACTION IS 1301, Public Library Deventer;
      %  Deventer Verhaal, Netherlands

      % //other
      % 7–9 December 2015 - Participation in the “Crossing borders. Te transmission
      %  of religious texts and the migration of scholars in late medieval and early
      %  modern Europe". COST Action IS1301 Training School at Budapest, Hungary

      % // other
      % 2015 - Part of the organizing group of a conference aimed at undergraduate and
      %  graduate students
      %  (FiKon, htp://m.cdn.blog.hu/f/fkon/image/felhivas\_fkon\_2015.pdf, only in
      %  Hungarian)

      % // other publications
      % 2015 - publication: Az 1540 előtti magyar ráolvasások: „vagyok bűnes [...] régti
      %  nagy sok bálványtimádásomba” [Hungartian tincantattions before 1540] in
      %  Információtörténeti Műhely, ed. Debóra BALÁZS, Zsófa Ágnes BARTÓK, Péter
      %  BOGNÁR, Szilvia MARÓTHY, Budapest, ELTE BTK Régi magyar irodalom
      %  Tanszék, 66–90, 2015, (htp://renaissance.elte.hu/wp-
      %  content/uploads/2016/02/gremlek.pdf, only in Hungarian)

      % // other publication
      % 2013 - publication: A Szelesteti-féle ráolvasás (The tincantattion "Szelesteti") in
      %  VERS: Verstan, poéttika, trópusok a 15–17. századti Európában, Fiatalok
      %  Konferenciája 2013, Budapest, reciti, 93–105. (htp://reciti.hu/wp-
      %  content/uploads/fkon1vn.pdf, only in Hungarian)


\end{document}