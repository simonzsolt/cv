\documentclass{article}

\begin{document}

CV

Zsolt Simon

Születési dátum: 1989 OKtóber 24

Cím: Magyarország, 2600, Vác, Erzsébet utca 10/A

E-mail: simon.zsolt@mail.com

Telefon: +36 20 4110 724

Tanulmányok

University: Eötvös Loránd University of Budapest, Faculty of Humanities
      From 2015 ongoing studies (Master's degree) at Te Center for Renaissance
       Studies (CHER - Centre des Hautes Études de la Renaissance) at Eötvös Loránd
       University of Budapest
      2015 Diploma (Bachelor's degree in Hungarian) at Eötvös Loránd University of
       Budapest
      Research director and topic: Iván Horváth, in studying medieval literary
       history and textual studies applied to the methods of digital humanities
       (database of old Hungarian verse, online critical editions)
Employment and other academic activities
      From 2017 - Contributor in „POSTDATA – Poetry Standardization and Linked
       Open Data” project. I presented the datamodell for the database of 16 century
                                                                            th

       Hungarian poems (Répertoire de la poésie hongroise ancienne, RPHA,
       htp://rpha.elte.hu/)

      26–28 April 2017 - Participation in the "Late Medieval and Early Modern Book
       Culture - Book Production, Book Ownership, and Book Collecting"- 5th
       TRAINING SCHOOL COST ACTION IS 1301, Public Library Deventer;
       Deventer Verhaal, Netherlands

      From 2016 - Participant in a project to create an online critical edition of a
       medieval hungarian book (Codex Pray) containing the frst piece of Hungarian
       literature (Halotti beszéd és könyörgés [Funeral Sermon and Prayer], HBK). My
       duties will be concerned with digital philology.




                                             1
Zsolt Simon                                                                           CV

      17 October 2016 - Presentation at "Pesti Bölcsész Akadémia" conference series
       held at Eötvös Loránd University of Budapest, Faculty of Humanities,
       Gráfadatbáztisok az tirodalomtudományban [Graph databases tin ltiterary studties],
       (htp://pestibolcseszakademia.blog.hu/2016/10/04/digitalis\_bolcseszet\_szovegen\_innen\_es\_tul\_marothy\_szilvia\_es\_simon\_zsolt\_kurzusa)

      24–28 October 2016 - Participation in DARIAH Winter School “Open Data
       Citation for Social Science and Humanities”, Humanities at Scale project (HaS),
       Prague

      7 November 2016 - Presentation at "Pesti Bölcsész Akadémia" conference series
       held at Eötvös Loránd University of Budapest, Faculty of Humanities, Hálózati
       versadatbázisok [Online databases of verses],
       (htp://pestibolcseszakademia.blog.hu/2016/10/04/digitalis\_bolcseszet\_szovegen
       \_innen\_es\_tul\_marothy\_szilvia\_es\_simon\_zsolt\_kurzusa)

      1–6 February 2016 - Participation in the "Paleography, Codicology, Philology.
       Digital Editing of Medieval Manuscripts Training Programme" at
       Klosterneuburg, Vienna
      2015–2016 - Technical editor of the journal Informáctiótörténetti Műhely
       [Informattion Htistory Workshop] for the following editions:

        Péter BOGNÁR, A régti magyar párrímköltészet német vonatkozásati, 2016,
         (htp://renaissance.elte.hu/wp-content/uploads/2016/02/parrim.pdf)

        Iván HORVÁTH, Ómagyar szövegemlékek mtint textológtiati tárgyak, 2015,
         (htp://renaissance.elte.hu/wp-content/uploads/2016/01/Omagyar\_szo%CC
         %88vegemle%CC%81kek\_d-ed-150bb.pdf)

        Tiziano T UBAY, A székely írás kutatásának története, 2015,
         (htp://renaissance.elte.hu/wp-content/uploads/2016/04/Tubay\_A-szekely-
         iras-kutatasanak-tortenete\_OSzK\_2015\_digitalis.pdf)
      7–9 December 2015 - Participation in the “Crossing borders. Te transmission
       of religious texts and the migration of scholars in late medieval and early
       modern Europe". COST Action IS1301 Training School at Budapest, Hungary
      24 November 2015 - Presentation at the "DHU2015 Workshop, Számítógép az
       irodalomtudományban" [Computers in Literary Studies], held by the Library of
       the Hungarian Academy of Sciences and Budapest University of Technology
       and Economics, (htp://dhu2015.mit.bme.hu/felhivas, only in Hungarian)



                                           2
Zsolt Simon                                                                                 CV

      2015 - Part of the organizing group of a conference aimed at undergraduate and
       graduate students
       (FiKon, htp://m.cdn.blog.hu/f/fkon/image/felhivas\_fkon\_2015.pdf, only in
       Hungarian)
      2015 development of the databse concerning medieval Hungarian book culture
       (book collections between 1000–1526 in the Kingdom of Hungary) known as
       THECA (htp://hece.elte.hu/index.php/theca/, htps://theca-
       online.herokuapp.com/\#/list)
      2015 - publication: Az 1540 előtti magyar ráolvasások: „vagyok bűnes [...] régti
       nagy sok bálványtimádásomba” [Hungartian tincantattions before 1540] in
       Információtörténeti Műhely, ed. Debóra BALÁZS, Zsófa Ágnes BARTÓK, Péter
       BOGNÁR, Szilvia MARÓTHY, Budapest, ELTE BTK Régi magyar irodalom
       Tanszék, 66–90, 2015, (htp://renaissance.elte.hu/wp-
       content/uploads/2016/02/gremlek.pdf, only in Hungarian)
      2014 - Collaborator at the Digital Philology Division at National Széchényi
       Library
      2014 - Developer of the Régti Magyar Exemplumadatbáztis (Database of Old
       Hungartian Exempla, htp://sermones.elte.hu/exemplumadatbazis/)
      2013 - publication: A Szelesteti-féle ráolvasás (Te tincantattion "Szelesteti") in
       VERS: Verstan, poéttika, trópusok a 15–17. századti Európában, Fiatalok
       Konferenciája 2013, Budapest, reciti, 93–105. (htp://reciti.hu/wp-
       content/uploads/fkon1vn.pdf, only in Hungarian)
      2013 - Junior editor of the book Mathtias Rex 1458–1490: Hungary at the Dawn
       of the Renatissance, (htp://renaissance.elte.hu/?page\_id=452)
      2013–2016 - Internship at the Department of Early Hungarian Literature,
       Institute of Hungarian Literature and Cultural Studies at Eötvös Loránd
       University of Budapest
Languages
      Hungarian - native
      English - high profciency in reading and writing (C1)
      Latin - elementary profciency in reading
      Italian - elementary profciency in reading




                                              3
Zsolt Simon                                                      CV

Technical skills
      PHP
      JavaScript development
        ES6
        backend: Node.js
        frontend: Angular, Material, Bootstrap
        RESTful APIs
        Singe Page Applications
      Unit testing: Mocha, Jasmin
      Git
      XML (TEI)
      HTTP protocol
      Databases, query languages:
        MySQL and derivatives
        SQL and derivatives
        NoSQL databases:
              Document databases: MongoDB, ElasticDB, CouchDB
              Graph databases: Neo4j
              Multimodel databses: ArangoDB, OrientDB
      experience with UNIX-like environments
      typeseting: LATEΧ




                                        4


\end{document}